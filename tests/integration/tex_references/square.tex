\documentclass[11pt]{article}
\usepackage{geometry}
\usepackage[american]{babel}
\usepackage{amsmath}
\usepackage{amssymb}
\usepackage[hidelinks]{hyperref}
\usepackage{tabularx}
\usepackage{ltablex}
\keepXColumns

\begin{document}
\section*{Symbols}


\subsection*{Sets}
\begin{tabularx}{\textwidth}{| l | l | X |}
\hline
\textbf{Name} & \textbf{Domains} & \textbf{Description}\\
\hline
\endhead

i & * & corner points of square\\
\hline
\end{tabularx}
\subsection*{Parameters}
\begin{tabularx}{\textwidth}{| l | l | X |}
\hline
\textbf{Name} & \textbf{Domains} & \textbf{Description}\\
\hline
\endhead

\hline
\end{tabularx}
\subsection*{Variables}
\begin{tabularx}{\textwidth}{| l | l | X |}
\hline
\textbf{Name} & \textbf{Domains} & \textbf{Description}\\
\hline
\endhead

t & i & position of square corner points on curve\\
x &  & x-coordinate of lower-left corner of square (=fx(t('1')))\\
y &  & y-coordinate of lower-left corner of square (=fy(t('1')))\\
a &  & horizontal distance between lower-left and lower-right corner of square\\
b &  & vertical distance between lower-left and lower-right corner of square\\
\hline
\end{tabularx}
\subsection*{Equations}
\begin{tabularx}{\textwidth}{| l | l | X |}
\hline
\textbf{Name} & \textbf{Domains} & \textbf{Description}\\
\hline
\endhead

e1x &  & define x-coordinate of lower-left corner\\
e1y &  & define y-coordinate of lower-left corner\\
e2x &  & define x-coordinate of lower-right corner\\
e2y &  & define y-coordinate of lower-right corner\\
e3x &  & define x-coordinate of upper-left corner\\
e3y &  & define y-coordinate of upper-left corner\\
e4x &  & define x-coordinate of upper-right corner\\
e4y &  & define y-coordinate of upper-right corner\\
\hline
\end{tabularx}
\subsection*{Model Definition}
\textbf{max} $(( power(a,2) ) + ( power(b,2) ))$\\
\textbf{s.t.}
\subsubsection*{$e1x$}
$
((( sin(t("1")) ) \cdot ( cos((t("1") - (t("1") * t("1")))) )) = x)
$
\vspace{5pt}
\hrule
\subsubsection*{$e1y$}
$
((t_\text{\textquotesingle 1\textquotesingle} \cdot ( sin(t("1")) )) = y)
$
\vspace{5pt}
\hrule
\subsubsection*{$e2x$}
$
((( sin(t("2")) ) \cdot ( cos((t("2") - (t("2") * t("2")))) )) = (x + a))
$
\vspace{5pt}
\hrule
\subsubsection*{$e2y$}
$
((t_\text{\textquotesingle 2\textquotesingle} \cdot ( sin(t("2")) )) = (y + b))
$
\vspace{5pt}
\hrule
\subsubsection*{$e3x$}
$
((( sin(t("3")) ) \cdot ( cos((t("3") - (t("3") * t("3")))) )) = (x - b))
$
\vspace{5pt}
\hrule
\subsubsection*{$e3y$}
$
((t_\text{\textquotesingle 3\textquotesingle} \cdot ( sin(t("3")) )) = (y + a))
$
\vspace{5pt}
\hrule
\subsubsection*{$e4x$}
$
((( sin(t("4")) ) \cdot ( cos((t("4") - (t("4") * t("4")))) )) = ((x + a) - b))
$
\vspace{5pt}
\hrule
\subsubsection*{$e4y$}
$
((t_\text{\textquotesingle 4\textquotesingle} \cdot ( sin(t("4")) )) = ((y + a) + b))
$
\vspace{5pt}
\hrule
\bigskip
$a\geq 0 ~ \forall \\$
$b\geq 0 ~ \forall \\$
\end{document}