\documentclass[11pt]{article}
\usepackage{geometry}
\usepackage[american]{babel}
\usepackage{amsmath}
\usepackage{amssymb}
\usepackage[hidelinks]{hyperref}
\usepackage{tabularx}
\usepackage{ltablex}
\keepXColumns
\usepackage{xcolor}
\setlength{\parindent}{0pt}

\begin{document}
\section*{Symbols}


\subsection*{Sets}
\begin{tabularx}{\textwidth}{| l | l | X |}
\hline
\textbf{Name} & \textbf{Domains} & \textbf{Description}\\
\hline
\endhead

i & * & corner points of square\\
\hline
\end{tabularx}
\subsection*{Parameters}
\begin{tabularx}{\textwidth}{| l | l | X |}
\hline
\textbf{Name} & \textbf{Domains} & \textbf{Description}\\
\hline
\endhead

\hline
\end{tabularx}
\subsection*{Variables}
\begin{tabularx}{\textwidth}{| l | l | X |}
\hline
\textbf{Name} & \textbf{Domains} & \textbf{Description}\\
\hline
\endhead

t & i & position of square corner points on curve\\
x &  & x-coordinate of lower-left corner of square (=fx(t('1')))\\
y &  & y-coordinate of lower-left corner of square (=fy(t('1')))\\
a &  & horizontal distance between lower-left and lower-right corner of square\\
b &  & vertical distance between lower-left and lower-right corner of square\\
square\_objective\_variable &  & \\
\hline
\end{tabularx}
\subsection*{Equations}
\begin{tabularx}{\textwidth}{| l | l | X |}
\hline
\textbf{Name} & \textbf{Domains} & \textbf{Description}\\
\hline
\endhead

e1x &  & define x-coordinate of lower-left corner\\
e1y &  & define y-coordinate of lower-left corner\\
e2x &  & define x-coordinate of lower-right corner\\
e2y &  & define y-coordinate of lower-right corner\\
e3x &  & define x-coordinate of upper-left corner\\
e3y &  & define y-coordinate of upper-left corner\\
e4x &  & define x-coordinate of upper-right corner\\
e4y &  & define y-coordinate of upper-right corner\\
square\_objective &  & \\
\hline
\end{tabularx}
\section*{Equation Definitions}
\subsubsection*{$e1x$}
\begin{equation}
 sin(t("1"))  \cdot  cos((t("1") - (t("1") * t("1"))))  = x
\end{equation}
\vspace{5pt}
\hrule
\subsubsection*{$e1y$}
\begin{equation}
t_{*} \cdot  sin(t("1"))  = y
\end{equation}
\vspace{5pt}
\hrule
\subsubsection*{$e2x$}
\begin{equation}
 sin(t("2"))  \cdot  cos((t("2") - (t("2") * t("2"))))  = x + a
\end{equation}
\vspace{5pt}
\hrule
\subsubsection*{$e2y$}
\begin{equation}
t_{*} \cdot  sin(t("2"))  = y + b
\end{equation}
\vspace{5pt}
\hrule
\subsubsection*{$e3x$}
\begin{equation}
 sin(t("3"))  \cdot  cos((t("3") - (t("3") * t("3"))))  = x - b
\end{equation}
\vspace{5pt}
\hrule
\subsubsection*{$e3y$}
\begin{equation}
t_{*} \cdot  sin(t("3"))  = y + a
\end{equation}
\vspace{5pt}
\hrule
\subsubsection*{$e4x$}
\begin{equation}
 sin(t("4"))  \cdot  cos((t("4") - (t("4") * t("4"))))  = x + a - b
\end{equation}
\vspace{5pt}
\hrule
\subsubsection*{$e4y$}
\begin{equation}
t_{*} \cdot  sin(t("4"))  = y + a + b
\end{equation}
\vspace{5pt}
\hrule
\subsubsection*{$square\_objective$}
\begin{equation}
 power(a,2)  +  power(b,2)  = square\_objective\_variable
\end{equation}
\vspace{5pt}
\hrule
\bigskip
$a\geq 0 ~ \forall \\$
$b\geq 0 ~ \forall \\$
\end{document}